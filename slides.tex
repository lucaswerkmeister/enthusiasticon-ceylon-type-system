\documentclass[aspectratio=169]{beamer}
\usepackage[utf8]{inputenc}
\usepackage{hyperref}
\usepackage{listings}
\usepackage{xcolor}

\usetheme{Madrid}
\beamertemplatenavigationsymbolsempty
\setbeamertemplate{footline}[text line]{%
  \parbox{0.3\linewidth}{%
    \insertshortauthor%
  }%
  \hfill%
  \parbox{0.3\linewidth}{%
    \url{https://goo.gl/zCbUxi}%
  }%
  \hfill%
  \raggedleft\insertpagenumber%
}

\title{The Ceylon Type System}
\author{Lucas Werkmeister}
\date{EnthusiastiCon 2018}

\lstdefinelanguage{ceylon}{
  morekeywords={
    assembly, module, package, import,
    alias, class, interface, object, given, value, assign, void, function, new,
    of, extends, satisfies, abstracts,
    in, out,
    return, break, continue, throw,
    assert, dynamic,
    if, else, switch, case, for, while, try, catch, finally, then, let,
    this, outer, super,
    is, exists, nonempty
  },
  morekeywords=[2]{
    shared, abstract, formal, default, actual, variable, late, native, deprecated, final, sealed, annotation, suppressWarnings, small, static
  },
  morekeywords=[3]{
    doc, by, license, see, throws, tagged
  },
  sensitive=true,
  morecomment=[l]{//},
  morecomment=[n]{/*}{*/},
  morestring=[b]",
  morestring=[b]',
}

% all colors taken from plugins/org.eclipse.ceylon.ide.eclipse.ui/plugin.xml in the ceylon-ide-eclipse repository
\definecolor{org.eclipse.ceylon.ide.eclipse.ui.theme.color.keywords}{RGB}{76,76,76}
\definecolor{org.eclipse.ceylon.ide.eclipse.ui.theme.color.annotations}{RGB}{51,153,204}
\definecolor{org.eclipse.ceylon.ide.eclipse.ui.theme.color.identifiers}{RGB}{0,51,153}
\definecolor{org.eclipse.ceylon.ide.eclipse.ui.theme.color.types}{RGB}{153,0,102}
\colorlet{org.eclipse.ceylon.ide.eclipse.ui.theme.color.numbers}{blue}
\colorlet{org.eclipse.ceylon.ide.eclipse.ui.theme.color.strings}{blue}
\colorlet{org.eclipse.ceylon.ide.eclipse.ui.theme.color.comments}{darkgray}
% https://tex.stackexchange.com/a/4199/25264
\makeatletter
\newcommand*\ceylonIdeEclipseIdentifier{%
  \expandafter\ceylonIdeEclipseIdentifier@style\the\lst@token\relax%
}
\def\ceylonIdeEclipseIdentifier@style#1#2\relax{%
  \ifcat#1%
    \relax%
  \else%
    \ifnum`#1=\uccode`#1%
      \color{org.eclipse.ceylon.ide.eclipse.ui.theme.color.types}%
    \else%
      \color{org.eclipse.ceylon.ide.eclipse.ui.theme.color.identifiers}%
    \fi%
  \fi%
}
\makeatother
\lstdefinestyle{org.eclipse.ceylon.ide.eclipse.ui.theme}{
  basicstyle=\ttfamily,
  keywordstyle=\bfseries\color{org.eclipse.ceylon.ide.eclipse.ui.theme.color.keywords},
  keywordstyle=[2]\color{org.eclipse.ceylon.ide.eclipse.ui.theme.color.annotations},
  keywordstyle=[3]\color{org.eclipse.ceylon.ide.eclipse.ui.theme.color.annotations},
  identifierstyle=\ceylonIdeEclipseIdentifier,
  numberstyle=\color{org.eclipse.ceylon.ide.eclipse.ui.theme.color.numbers},
  stringstyle=\color{org.eclipse.ceylon.ide.eclipse.ui.theme.color.strings},
  commentstyle=\color{org.eclipse.ceylon.ide.eclipse.ui.theme.color.comments},
}

% all colors taken from typechecker/en/styles/ceylon.css in the ceylon repository
\definecolor{ceylon.rainbow.comment}{HTML}{008b8b} % darkcyan
\colorlet{ceylon.rainbow.constant.numeric}{blue}
\colorlet{ceylon.rainbow.constant.string}{blue}
\definecolor{ceylon.rainbow.entity.function}{RGB}{51,153,204}
\definecolor{ceylon.rainbow.entity.class}{RGB}{153,0,102}
\definecolor{ceylon.rainbow.variable.global}{RGB}{0,51,153}
\definecolor{ceylon.rainbow.keyword}{RGB}{76,76,76}
% https://tex.stackexchange.com/a/4199/25264
\makeatletter
\newcommand*\ceylonRainbowIdentifier{%
  \expandafter\ceylonRainbowIdentifier@style\the\lst@token\relax%
}
\def\ceylonRainbowIdentifier@style#1#2\relax{%
  \ifcat#1%
    \relax%
  \else%
    \ifnum`#1=\uccode`#1%
      \color{ceylon.rainbow.entity.class}%
    \else%
      \color{ceylon.rainbow.variable.global}%
    \fi%
  \fi%
}
\makeatother
\lstdefinestyle{ceylon.rainbow}{ % mostly identical to org.eclipse.ceylon.ide.eclipse.ui.theme – see also https://github.com/eclipse/ceylon/issues/7343
  basicstyle=\ttfamily,
  keywordstyle=\bfseries\color{ceylon.rainbow.keyword},
  keywordstyle=[2]\color{ceylon.rainbow.entity.function},
  keywordstyle=[3]\color{ceylon.rainbow.entity.function},
  identifierstyle=\ceylonRainbowIdentifier,
  numberstyle=\color{ceylon.rainbow.constant.numeric},
  stringstyle=\color{ceylon.rainbow.constant.string},
  commentstyle=\color{ceylon.rainbow.comment},
}

% all colors taken from stylesheets/themes/paraiso-dark.css in the ceylon-lang.org repository
\definecolor{ceylon.paraiso.background}{HTML}{2f1e2e}
\definecolor{ceylon.paraiso.foreground}{HTML}{e7e9db}
\definecolor{ceylon.paraiso.comment}{HTML}{776e71}
\definecolor{ceylon.paraiso.constant.numeric}{HTML}{f99b15}
\definecolor{ceylon.paraiso.constant.string}{HTML}{48b685}
\definecolor{ceylon.paraiso.entity.function}{HTML}{06b6ef}
\definecolor{ceylon.paraiso.entity.class}{HTML}{fec418}
\definecolor{ceylon.paraiso.variable.global}{HTML}{ef6155}
\definecolor{ceylon.paraiso.keyword}{HTML}{b8baaf}
% https://tex.stackexchange.com/a/4199/25264
\makeatletter
\newcommand*\ceylonParaisoIdentifier{%
  \expandafter\ceylonParaisoIdentifier@style\the\lst@token\relax%
}
\def\ceylonParaisoIdentifier@style#1#2\relax{%
  \ifcat#1%
    \relax%
  \else%
    \ifnum`#1=\uccode`#1%
      \color{ceylon.paraiso.entity.class}%
    \else%
      \color{ceylon.paraiso.variable.global}%
    \fi%
  \fi%
}
\makeatother
\lstdefinestyle{ceylon.paraiso}{ % WARNING: do not use with \lstinline – \lstinline does not apply the backgroundcolor and this theme is all but unreadable against a light background
  basicstyle=\ttfamily\color{ceylon.paraiso.foreground},
  backgroundcolor=\color{ceylon.paraiso.background},
  keywordstyle=\color{ceylon.paraiso.keyword},
  keywordstyle=[2]\color{ceylon.paraiso.entity.function},
  keywordstyle=[3]\color{ceylon.paraiso.entity.function},
  identifierstyle=\ceylonParaisoIdentifier,
  numberstyle=\color{ceylon.paraiso.constant.numeric},
  stringstyle=\color{ceylon.paraiso.constant.string},
  commentstyle=\color{ceylon.paraiso.comment},
}

\lstset{language=ceylon}
%\lstset{style=org.eclipse.ceylon.ide.eclipse.ui.theme}
\lstset{style=ceylon.rainbow}

\newcommand{\TwitterUsername}[1]{\href{https://twitter.com/#1}{\ttfamily @#1}}

\begin{document}

\frame{\titlepage}

\begin{frame}[fragile]
  \frametitle{Type systems}
  \begin{onlyenv}<1>
    \begin{lstlisting}
function timesTwo(num) {
  return num * 2;
}
    \end{lstlisting}
  \end{onlyenv}
  \begin{onlyenv}<2>
    \begin{lstlisting}
Float timesTwo(Float num) {
  return num * 2;
}
    \end{lstlisting}
  \end{onlyenv}
  \begin{onlyenv}<3>
    \begin{lstlisting}
Float timesTwo(Float num) {
  return num << 1;
}
    \end{lstlisting}
  \end{onlyenv}
  \begin{onlyenv}<4>
    \begin{lstlisting}
Float timesTwo(Float num) {
  return num << 1; // error: only ints support bitshift
}
    \end{lstlisting}
  \end{onlyenv}
\end{frame}

\begin{frame}[fragile]
  \frametitle{Union types}
  A value of type \lstinline{X|Y} can have the type \lstinline{X} or \lstinline{Y}.
  \pause
  \begin{lstlisting}
shared Document|ParseError parseDocument(String html);
  \end{lstlisting}
  \pause
  \begin{lstlisting}
shared void enqueue(Element item, Integer? priority = null);
  \end{lstlisting}
  \pause
  \lstinline{Integer?} is equivalent to \lstinline{Integer|Null}.
  (\lstinline{Null} is a class with only one instance, \lstinline{null}.)
\end{frame}

\begin{frame}[fragile]
  \frametitle{Intersection types}
  A value of type \lstinline{X&Y} has the types \lstinline{X} and \lstinline{Y}.
  \pause
  \begin{lstlisting}
shared void printScaled(
    Printable&Scalable element, Float factor);
  \end{lstlisting}
\end{frame}

\begin{frame}[fragile]
  \frametitle{Flow-sensitive typing}
  Instead of Java’s
  \begin{lstlisting}[language=java]
Object o = getSomething();
if (o instanceof String) {
  String s = (String)o;
  String su = s.toUpperCase();
}
  \end{lstlisting}
  \pause
  you simply write
  \begin{lstlisting}
Object o = getSomething();
if (is String o) {
  // o has type String now
  String su = o.uppercased; // look ma, no cast!
}
  \end{lstlisting}
\end{frame}

\begin{frame}[fragile]
  \frametitle{Flow-sensitive typing}
  \begin{onlyenv}<1>
    \begin{lstlisting}
String? arg = process.arguments.first;
if (exists arg) {
  // arg: String





  print(arg.uppercased);
}
    \end{lstlisting}
  \end{onlyenv}
  \begin{onlyenv}<2>
    \begin{lstlisting}
String? arg = process.arguments.first;
if (exists arg) { // is Object arg
  // arg: String





  print(arg.uppercased);
}
    \end{lstlisting}
  \end{onlyenv}
  \begin{onlyenv}<3>
    \begin{lstlisting}
String? arg = process.arguments.first;
if (exists arg) { // is Object arg
  // arg: <String?>&Object





  print(arg.uppercased);
}
    \end{lstlisting}
  \end{onlyenv}
  \begin{onlyenv}<4>
    \begin{lstlisting}
String? arg = process.arguments.first;
if (exists arg) { // is Object arg
  // arg: <String?>&Object
  //    = <String|Null>&Object




  print(arg.uppercased);
}
    \end{lstlisting}
  \end{onlyenv}
  \begin{onlyenv}<5>
    \begin{lstlisting}[escapechar=!]
String? arg = process.arguments.first;
if (exists arg) { // is Object arg
  // arg: <String?>&Object
  //    = <String|Null>&Object !$(\text{String} \cup \text{Null}) \cap \text{Object}$!




  print(arg.uppercased);
}
    \end{lstlisting}
  \end{onlyenv}
  \begin{onlyenv}<6>
    \begin{lstlisting}[escapechar=!]
String? arg = process.arguments.first;
if (exists arg) { // is Object arg
  // arg: <String?>&Object
  //    = <String|Null>&Object !$(\text{String} \cup \text{Null}) \cap \text{Object}$!
                               !$(\text{String}\cap\text{Object}) \cup (\text{Null}\cap\text{Object})$!



  print(arg.uppercased);
}
    \end{lstlisting}
  \end{onlyenv}
  \begin{onlyenv}<7>
    \begin{lstlisting}
String? arg = process.arguments.first;
if (exists arg) { // is Object arg
  // arg: <String?>&Object
  //    = <String|Null>&Object
  //    = <String&Object>|<Null&Object>



  print(arg.uppercased);
}
    \end{lstlisting}
  \end{onlyenv}
  \begin{onlyenv}<8>
    \begin{lstlisting}
String? arg = process.arguments.first;
if (exists arg) { // is Object arg
  // arg: <String?>&Object
  //    = <String|Null>&Object
  //    = <String&Object>|<Null&Object>
  //    = String|<Null&Object>


  print(arg.uppercased);
}
    \end{lstlisting}
  \end{onlyenv}
  \begin{onlyenv}<9>
    \begin{lstlisting}
String? arg = process.arguments.first;
if (exists arg) { // is Object arg
  // arg: <String?>&Object
  //    = <String|Null>&Object
  //    = <String&Object>|<Null&Object>
  //    = String|<Null&Object>
  //    = String|Nothing

  print(arg.uppercased);
}
    \end{lstlisting}
  \end{onlyenv}
  \begin{onlyenv}<10>
    \begin{lstlisting}
String? arg = process.arguments.first;
if (exists arg) { // is Object arg
  // arg: <String?>&Object
  //    = <String|Null>&Object
  //    = <String&Object>|<Null&Object>
  //    = String|<Null&Object>
  //    = String|Nothing
  //    = String
  print(arg.uppercased);
}
    \end{lstlisting}
  \end{onlyenv}
\end{frame}

\begin{frame}
  \frametitle{Thank you!}
  \begin{itemize}
  \item If this sounded interesting, please talk to me later or keep in touch!
  \item \TwitterUsername{LucasWerkmeistr} or \TwitterUsername{ceylonlang}
  \item \url{https://ceylon-lang.org}
    \begin{itemize}
    \item \url{https://ceylon-lang.org/documentation/current/introduction/}
    \item \url{https://ceylon-lang.org/documentation/current/tour/}
    \end{itemize}
  \end{itemize}
\end{frame}

\begin{frame}
  \frametitle{Backup slide!}
  More cool stuff!
  \begin{itemize}
  \item Declaration-site variance: \lstinline{Set<Integer>|Set<Float> = Set<Integer|Float>}
  \item Tuple types like \lstinline{[String, Integer, Integer]} are linked lists of types (arbitrary length)
  \item Tuple types are used for function types like \lstinline{Integer(Integer, Integer)} (no \lstinline{Function22} limit like in Scala)
  \end{itemize}
\end{frame}

\end{document}
